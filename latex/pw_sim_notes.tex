\documentclass{article}
\usepackage{amsmath}


\newcommand{\x}{\mathbf{x}}

\begin{document}

\title{Pilot wave simulation notes}
\author{Kamal Ndousse}
\date{\today}

\maketitle

\section{Discretization}

According to [Bush's review paper], the field created by a walking droplet is of the form:
\begin{equation}
	\label{eq:raw_field}
	h(\x,t) = A \sum_{n = -\infty}^{\left[ t/T_{F} \right]} J_{0} \left( K_{F} \left| \x - \x_{p}(n T_{F}) \right| \right) e^{-(t - n T_{F})/(M T_{F})},
\end{equation}
and the position of the droplet is determined by
\begin{equation}
	\label{eq:raw_droplet}
	m \ddot{\x_p} + D \dot{\x_p} = - m g \nabla h(\x_p,t).
\end{equation}

\subsection{Field discretization}
The field at time $t + T_{F}$ is
\begin{align}
	h(\x,t+T_F) &= A \sum_{n = -\infty}^{\left[ t/T_{F} \right]+1} J_{0} \left( K_{F} \left| \x - \x_{p}(n T_{F}) \right| \right) e^{-(t +T_F- n T_{F})/(M T_{F})} \notag \\
	 &= A \sum_{n = -\infty}^{\left[ t/T_{F} \right]} J_{0} \left( K_{F} \left| \x - \x_{p}(n T_{F}) \right| \right) e^{-(t- n T_{F})/(M T_{F})} e^{-1/M} \notag \\
	 & + A J_{0} \left( K_F \left| \x - \x_p(t + T_F) \right| \right) \notag \\
	h(\x,t+T_F) &= A J_0 \left( K_F \left| \x - \x_p(t + T_F) \right | \right) + e^{-1/M} h(\x,t) \label{eq:discrete_field}
\end{align}

\subsection{Discrete particle motion}
Similarly we can express the position of the particle at time $t + T_{F}$ using the centered second-difference approximation for the second derivative
\begin{equation}
	\ddot{\x_p}(t) \approx \frac{\x_p(t + T_F) - 2 \x_p(t) + \x_p(t - T_F)}{T_F^2}
\end{equation}
and Sterling's formula for centered differences for the first derivative
\begin{equation}
	\dot{\x_p}(t) \approx \frac{\x_p(t + T_F) - \x_p(t - T_F)}{2 T_F}.
\end{equation}
Plugging these into Eq.~\ref{eq:raw_droplet} gives:
\begin{equation}
	\label{eq:dd_crappy}
	-g \nabla h(\x_p(t),t) = \frac{\x_p(t + T_F) - 2 \x_p(t) + \x_p(t - T_F)}{T_F^2}
	 + \frac{D}{m} \frac{\x_p(t + T_F) - \x_p(t - T_F)}{2 T_F}.
\end{equation}
Solving for $\x_p(t + T_F)$, we find:
\begin{equation}
	\label{eq:dd_crappy_2}
	\x_p(t + T_F) = \left( \frac{4 m}{2 m + D T_F} \right) \x_p(t) + \left( \frac{D T_F - 2m}{D T_F + 2m} \right) \x_p(t - T_F) - \left( \frac{2 T_F^2 g}{D T_f + 2m} \right) \nabla h(\x_p(t),t).
\end{equation}

\subsection{Nondimensionalization}
The expressions giving the discrete time evolution of the field (Eq.~\ref{eq:discrete_field}) and droplet (Eq.~\ref{eq:dd_crappy_2}) can be simplified by introducing the following non-dimensional variables:
\begin{align}
	Q &\equiv \frac{D T_F}{2 M}\text{, and }\\
	S &\equiv T_F^2 g A K_{F}.
\end{align}
$Q$ encapsulates properties which could vary between droplets on a single field (in particular, mass and drag), and $S$ accounts for general properties of the wave field (amplitide of bounces, gravitational force, faraday wave number).

In terms of these parameters, Eqs.~\ref{eq:discrete_field} and~\ref{eq:dd_crappy_2} become:
\begin{align}
	h(\x,t+T_F) &= S J_0 \left( K_F \left| \x - \x_p(t + T_F) \right | \right) + e^{-1/M} h(\x,t) \label{eq:nondim_field} \\
	\x_p(t + T_F) &= \left( \frac{2}{Q+1} \right) \x_p(t) + \left( \frac{Q-1}{Q+1} \right) \x_p(t - T_F) - \left( \frac{1}{K_F(Q+1)} \right) \nabla h(\x_p(t),t) \label{eq:nondim_field}
\end{align}

Finally, we switch to more suggestive notation for discrete numerical computation by defining:
\begin{align}
	\x_p[n] &= \x_p(n T_F) \text{, and }\\
	h_n[\x] &= h(\x,n T_F)
\end{align}

We then have:
\begin{align}
	h_{n+1}[\x] &= e^{-1/M} h_n[\x] + S J_{0}(|\x - \x_p[n+1]|) \text{, with }\\
	x_p[n+1] &= \left( \frac{2}{Q+1} \right) \x_p[n] + \left( \frac{Q-1}{Q+1} \right) \x_p[n-1] - \frac{1}{Q+1} \nabla h_{n}[x].
\end{align}








\end{document}
